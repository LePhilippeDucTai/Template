\documentclass[12pt,a4paper]{report}
\usepackage{lipsum}
\usepackage{style_bpifrance}
%------------------Document----------------------------

\begin{document}

\titre{Document de Méthodologie du Modèle de Pilotage des fonds de garantie individuels}

\tableofcontents


\newpage
\part{Introduction et problématique de modélisation}

\chapter{Etapes de la simulation du passif d'un fonds de garantie}
\section{Section}

Lorem ipsum dolor sit amet, consectetur adipiscing elit. Donec nec fermentum augue. Integer id neque sit amet augue lacinia fringilla. Donec leo ipsum, dapibus vel orci et, viverra viverra sem. Morbi maximus neque ipsum, in vulputate libero porta a. Interdum et malesuada fames ac ante ipsum primis in faucibus. Nulla at libero arcu. 
o ipsum, dapibus vel orci et, viverra viverra sem. Morbi maximus neque ipsum, in vulputate libero porta a. Interdum et malesuada fames ac ante ipsum primis in faucibus. Nulla at libero arcu.


\section{Second Section}

\[
\boxedb{\int_{-\infty}^{+\infty} e^{-x^2/2}dx = \sqrt{2\pi}.}
\]

\[
\boxedb{Z_{i,j,t} = -\sqrt{\rho_j}X_t + \sqrt{1-\rho_j}\varepsilon_{i,j,t}}
\]

De plus, les taux de sinistralités sont modélisés par :
\[
S_{i,j,t} = H^{-1}_{\mu,\sigma}(\Phi(Z_{i,j,t}))
\]


\begin{figure}[htb]
\centering
  \includegraphics[width=1\textwidth]{figures/figure1.png}
 \centering
  {\textbf{\caption{Figure Description}}}\label{fig:1}
\end{figure}


\subsection{Subsection}

In pretium enim dui, quis ornare odio varius et. Nulla facilisi. Nunc tristique tortor at urna vehicula, elementum bibendum tortor hendrerit. Ut eu risus nisi. Vestibulum nunc lorem, tristique sed ultrices nec, porta et ligula. Nam facilisis felis a congue auctor. Fusce id odio in libero blandit cursus nec nec arcu. Nulla eu ultricies massa, id dapibus nisi. Donec tellus urna, maximus nec semper id, consectetur eu mi. Vestibulum elit eros, porta ac eros a, mattis pulvinar magna. Nulla pellentesque dapibus leo molestie varius. Duis ut rutrum urna, at condimentum dui. Sed orci magna, faucibus nec quam et, malesuada ultrices velit. Nam a lorem a massa facilisis rutrum eget id nunc. Morbi aliquet felis et tincidunt scelerisque. 

\begin{enumerate}
    \item Modélisation de la mise en jeu de la garantie
    \item Modélisation du taux de récupération
\end{enumerate}

\paragraph{Etape 1 : simulation de l'exposition globale probable}
A la date de simulation, l’exposition globale n’est pas connue, car certaines autorisations de garantie :

\begin{itemize}
    \item de la base de simulation ne sont pas encore utilisées, et pourraient l’être dans le futur
    \item 	ne sont pas encore injectées dans le système informatique. Elles sont donc absentes de la base de simulation. Le risque autorisé injecté avec retard est donc à estimer.
\end{itemize}


La simulation des utilisations futures de garanties non encore utilisées est la première étape de la simulation du passif d’un fonds de garantie. Cette simulation est faite en utilisant des taux et profil d’utilisation calibrés sur les données historiques par fonds de garantie.  Cette simulation des utilisations permet d’avoir l’exposition globale saine du fonds de garantie.

\paragraph{Etape 2 : simulation des remboursements anticipé et de la mise en jeu}

La deuxième étape est la simulation des sinistres ou des mises en jeu probables sur les expositions saines. La simulation des contentieux est la partie la plus importante dans l’estimation du passif d’un fonds de garantie. 
En général la mise en jeu de la garantie  a lieu :
\begin{itemize}
	\item si l’entreprise bénéficiaire du crédit garanti fait l’objet d’une procédure collective, dès le prononcé du jugement de redressement ou de liquidation judiciaire
 	\item dans le cas contraire, dès la notification de la résiliation du crédit, décidée d’un commun accord entre la banque et Bpifrance Financement.
\end{itemize}
 
Le montant total de mise en jeu (ou montant du sinistre) pour une maturité résiduelle maximale T des crédits et pour N entreprises bénéficiaires de crédits garantis, est donné comme suit:

\[
\boxedb{
Mt\_MeJ\_Ultime(N,T) = \sum_{t=1}^T\sum_{i=1}^N E_{t-1,i}\times\delta_{t,i}\times\indic{t\leq \tau_{RA_i}}
}
\]



\subsubsection{Subsubsection}
\lipsum

\begin{itemize}
    \item test
    \item testing
\item tester
\end{itemize}

\paragraph{Etape 1 : Simulation de la trajectoire future du facteur de risque systémique $Y_t$}
On utilise le modèle de dynamique pour $Y_t$ pour simuler différentes trajectoires futures $Y_t$  du facteur de risque jusqu’à la maturité résiduelle maximale des concours garantis. 

\subsubsection{Autre subsubsection}
Chaque valeur du facteur de risque est commune à toutes les entreprises

% %-------------------------Content------------------------
\subsection{Testing another subsection}

In pretium enim dui, quis ornare odio varius et. Nulla facilisi. Nunc tristique tortor at urna vehicula, elementum bibendum tortor hendrerit. Ut eu risus nisi. Vestibulum nunc lorem, tristique sed ultrices nec, porta et ligula. Nam facilisis felis a congue auctor. Fusce id odio in libero blandit cursus nec nec arcu. Nulla eu ultricies massa, id dapibus nisi. Donec tellus urna, maximus nec semper id, consectetur eu mi. Vestibulum elit eros, porta ac eros a, mattis pulvinar magna. Nulla pellentesque dapibus leo molestie varius. Duis ut rutrum urna, at condimentum dui. Sed orci magna, faucibus nec quam et, malesuada ultrices velit. Nam a lorem a massa facilisis rutrum eget id nunc. Morbi aliquet felis et tincidunt scelerisque. 

\paragraph{Lecture du tableau :} 
Prenons l’exemple du fonds de garantie Création et le cas des simulations au 30/06/2018. Notons par « S » le semestre de simulation (le 1er semestre 2018). Les montants utilisés par génération d’autorisation doivent être multipliés par :
\begin{itemize}
    \item 1.222 pour les autorisations du semestre « $S$ », c’est-à-dire de délai égal à 0 ;
    \item 1.047 pour les autorisations du semestre « $S -1$ », c’est-à-dire de délai égal à 1 ;
    \item 1.023 pour les autorisations du semestre « $S-2$ », c’est-à-dire de délai égal à 2.
\end{itemize}

\paragraph{}
In pretium enim dui, quis ornare odio varius et. Nulla facilisi. Nunc tristique tortor at urna vehicula, elementum bibendum tortor hendrerit. Ut eu risus nisi. Vestibulum nunc lorem, tristique sed ultrices nec, porta et ligula. Nam facilisis felis a congue auctor. Fusce id odio in libero blandit cursus nec nec arcu. Nulla eu ultricies massa, id dapibus nisi. Donec tellus urna, maximus nec semper id, consectetur eu mi.

\paragraph{Quelques remarques :}
\begin{itemize}
    \item La première méthode a principalement deux inconvénients :
    \begin{itemize}
        \item Le choix de la loi de probabilité est purement statistique. Elle ne se base sur aucune connaissance a priori du phénomène.
        \item La méthode ne garantit pas le choix d’une loi de probabilité qui modélise bien la queue de distribution. Or dans la modélisation quantitative du risque, la queue de distribution est très importante.
        \begin{itemize}
            \item Test bullet
            \item Test 2
        \end{itemize}
    \end{itemize}
    \item La deuxième méthode présente également quelques inconvénients : la loi choisie n’est pas statistiquement la loi la plus proche au vu des données historiques. Comment justifier le choix d’une loi plutôt qu’une autre ?
    \item Blabla
\end{itemize}

\section{Taux d'utilisation}

\newpage

\chapter{Modélisation de la mise en jeu d'une garantie}
Dans ce modèle individuel de pilotage des fonds de garantie de garantie, la simulation de la mise en jeu est une étape fondamentale. Elle peut être décomposée en deux parties :
\begin{itemize}
    \item la modélisation de la dépendance :
    \begin{itemize}
        \item entre les entreprises du portefeuille : choix de la structure de dépendance
        \item temporelle, permettant de prendre en compte des persistances des chocs macroéconomiques.
    \end{itemize}
    \item la modélisation de la mise en jeu: on va plutôt s’intéresser à avoir une estimation précise de la mise en jeu.
\end{itemize}

Pour modéliser la dépendance des risques au sein du portefeuille, nous avons retenu de l’évènement défaut, plutôt que la mise en jeu de la garantie. Cela permet de réduire le risque de biais éventuel dû à l’effet gestion.
La modélisation de la mise en permet de prendre en compte le fait que tous les défauts ne conduisent pas forcément à la mise en jeu.

\section{Modélisation de la dépendance}

Pour modéliser la dépendance, nous avons retenu un modèle à variables latentes. Dans ce modèle, on suppose que l’entreprise $i$ du fonds de garantie $j$ fait défaut si la variable $Z_{i,j,t}$ finie par la suite, passe en dessous d’un certain seuil $d_{i,j,t}$ déterministe) :
\[
\boxedb{Z_{i,j,t} = -\sqrt{\rho_j}X_t + \sqrt{1-\rho_j}\varepsilon_{i,j,t}}
\]

On fait les hypothèses suivantes :
\begin{itemize}
    \item $Y_t \sim \stdnorm$ : facteur de risque systémique commun à toutes les entreprises.
    \item $\varepsilon_{i,j,t}\sim\stdnorm$ : un facteur de risque idiosyncratique, propre à chaque entreprise.
    \item $\varepsilon_{i,j,t}$ et $Y_t$ indépendantes.
\end{itemize}

Pour prendre en compte la persistance des chocs macroéconomiques (ou l’autocorrélation des taux de défaut historiques), nous avons postulé le modèle suivant pour le facteur de risque systémique :
\[
\boxedb{Y_t = \sum_{k=1}^q \beta_k Y_{t-k} + \sigma\eta_t}
\]
\paragraph{Hypothèses}
\begin{itemize}
    \item $\mathbb{E}[Y_t] = 0$, $Var(Y_t) = 1, \forall t\geq 0$
    \item $\eta_t\sim \mathcal{N}(0,1)$
    \item $\forall k\geq 1$ les variables aléatoires $Y_{t-k}$ et $\eta_t$ sont indépendantes.
\end{itemize}

\end{document}
